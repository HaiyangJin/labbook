\documentclass[]{ctexbook}
\usepackage{lmodern}
\usepackage{amssymb,amsmath}
\usepackage{ifxetex,ifluatex}
\usepackage{fixltx2e} % provides \textsubscript
\ifnum 0\ifxetex 1\fi\ifluatex 1\fi=0 % if pdftex
  \usepackage[T1]{fontenc}
  \usepackage[utf8]{inputenc}
\else % if luatex or xelatex
  \ifxetex
    \usepackage{xltxtra,xunicode}
  \else
    \usepackage{fontspec}
  \fi
  \defaultfontfeatures{Ligatures=TeX,Scale=MatchLowercase}
\fi
% use upquote if available, for straight quotes in verbatim environments
\IfFileExists{upquote.sty}{\usepackage{upquote}}{}
% use microtype if available
\IfFileExists{microtype.sty}{%
\usepackage{microtype}
\UseMicrotypeSet[protrusion]{basicmath} % disable protrusion for tt fonts
}{}
\usepackage[b5paper,tmargin=2.5cm,bmargin=2.5cm,lmargin=3.5cm,rmargin=2.5cm]{geometry}
\usepackage[unicode=true]{hyperref}
\PassOptionsToPackage{usenames,dvipsnames}{color} % color is loaded by hyperref
\hypersetup{
            pdftitle={Face lab book},
            pdfauthor={Face lab},
            colorlinks=true,
            linkcolor=Maroon,
            citecolor=Blue,
            urlcolor=Blue,
            breaklinks=true}
\urlstyle{same}  % don't use monospace font for urls
\usepackage{natbib}
\bibliographystyle{apalike}
\usepackage{color}
\usepackage{fancyvrb}
\newcommand{\VerbBar}{|}
\newcommand{\VERB}{\Verb[commandchars=\\\{\}]}
\DefineVerbatimEnvironment{Highlighting}{Verbatim}{commandchars=\\\{\}}
% Add ',fontsize=\small' for more characters per line
\usepackage{framed}
\definecolor{shadecolor}{RGB}{248,248,248}
\newenvironment{Shaded}{\begin{snugshade}}{\end{snugshade}}
\newcommand{\AlertTok}[1]{\textcolor[rgb]{0.94,0.16,0.16}{#1}}
\newcommand{\AnnotationTok}[1]{\textcolor[rgb]{0.56,0.35,0.01}{\textbf{\textit{#1}}}}
\newcommand{\AttributeTok}[1]{\textcolor[rgb]{0.13,0.29,0.53}{#1}}
\newcommand{\BaseNTok}[1]{\textcolor[rgb]{0.00,0.00,0.81}{#1}}
\newcommand{\BuiltInTok}[1]{#1}
\newcommand{\CharTok}[1]{\textcolor[rgb]{0.31,0.60,0.02}{#1}}
\newcommand{\CommentTok}[1]{\textcolor[rgb]{0.56,0.35,0.01}{\textit{#1}}}
\newcommand{\CommentVarTok}[1]{\textcolor[rgb]{0.56,0.35,0.01}{\textbf{\textit{#1}}}}
\newcommand{\ConstantTok}[1]{\textcolor[rgb]{0.56,0.35,0.01}{#1}}
\newcommand{\ControlFlowTok}[1]{\textcolor[rgb]{0.13,0.29,0.53}{\textbf{#1}}}
\newcommand{\DataTypeTok}[1]{\textcolor[rgb]{0.13,0.29,0.53}{#1}}
\newcommand{\DecValTok}[1]{\textcolor[rgb]{0.00,0.00,0.81}{#1}}
\newcommand{\DocumentationTok}[1]{\textcolor[rgb]{0.56,0.35,0.01}{\textbf{\textit{#1}}}}
\newcommand{\ErrorTok}[1]{\textcolor[rgb]{0.64,0.00,0.00}{\textbf{#1}}}
\newcommand{\ExtensionTok}[1]{#1}
\newcommand{\FloatTok}[1]{\textcolor[rgb]{0.00,0.00,0.81}{#1}}
\newcommand{\FunctionTok}[1]{\textcolor[rgb]{0.13,0.29,0.53}{\textbf{#1}}}
\newcommand{\ImportTok}[1]{#1}
\newcommand{\InformationTok}[1]{\textcolor[rgb]{0.56,0.35,0.01}{\textbf{\textit{#1}}}}
\newcommand{\KeywordTok}[1]{\textcolor[rgb]{0.13,0.29,0.53}{\textbf{#1}}}
\newcommand{\NormalTok}[1]{#1}
\newcommand{\OperatorTok}[1]{\textcolor[rgb]{0.81,0.36,0.00}{\textbf{#1}}}
\newcommand{\OtherTok}[1]{\textcolor[rgb]{0.56,0.35,0.01}{#1}}
\newcommand{\PreprocessorTok}[1]{\textcolor[rgb]{0.56,0.35,0.01}{\textit{#1}}}
\newcommand{\RegionMarkerTok}[1]{#1}
\newcommand{\SpecialCharTok}[1]{\textcolor[rgb]{0.81,0.36,0.00}{\textbf{#1}}}
\newcommand{\SpecialStringTok}[1]{\textcolor[rgb]{0.31,0.60,0.02}{#1}}
\newcommand{\StringTok}[1]{\textcolor[rgb]{0.31,0.60,0.02}{#1}}
\newcommand{\VariableTok}[1]{\textcolor[rgb]{0.00,0.00,0.00}{#1}}
\newcommand{\VerbatimStringTok}[1]{\textcolor[rgb]{0.31,0.60,0.02}{#1}}
\newcommand{\WarningTok}[1]{\textcolor[rgb]{0.56,0.35,0.01}{\textbf{\textit{#1}}}}
\usepackage{longtable,booktabs}
% Fix footnotes in tables (requires footnote package)
\IfFileExists{footnote.sty}{\usepackage{footnote}\makesavenoteenv{long table}}{}
\IfFileExists{parskip.sty}{%
\usepackage{parskip}
}{% else
\setlength{\parindent}{0pt}
\setlength{\parskip}{6pt plus 2pt minus 1pt}
}
\setlength{\emergencystretch}{3em}  % prevent overfull lines
\providecommand{\tightlist}{%
  \setlength{\itemsep}{0pt}\setlength{\parskip}{0pt}}
\setcounter{secnumdepth}{5}
% Redefines (sub)paragraphs to behave more like sections
\ifx\paragraph\undefined\else
\let\oldparagraph\paragraph
\renewcommand{\paragraph}[1]{\oldparagraph{#1}\mbox{}}
\fi
\ifx\subparagraph\undefined\else
\let\oldsubparagraph\subparagraph
\renewcommand{\subparagraph}[1]{\oldsubparagraph{#1}\mbox{}}
\fi

% set default figure placement to htbp
\makeatletter
\def\fps@figure{htbp}
\makeatother

\usepackage{booktabs}
\usepackage{longtable}

\usepackage{framed,color}
\definecolor{shadecolor}{RGB}{248,248,248}

\renewcommand{\textfraction}{0.05}
\renewcommand{\topfraction}{0.8}
\renewcommand{\bottomfraction}{0.8}
\renewcommand{\floatpagefraction}{0.75}

\let\oldhref\href
\renewcommand{\href}[2]{#2\footnote{\url{#1}}}

\makeatletter
\newenvironment{kframe}{%
\medskip{}
\setlength{\fboxsep}{.8em}
 \def\at@end@of@kframe{}%
 \ifinner\ifhmode%
  \def\at@end@of@kframe{\end{minipage}}%
  \begin{minipage}{\columnwidth}%
 \fi\fi%
 \def\FrameCommand##1{\hskip\@totalleftmargin \hskip-\fboxsep
 \colorbox{shadecolor}{##1}\hskip-\fboxsep
     % There is no \\@totalrightmargin, so:
     \hskip-\linewidth \hskip-\@totalleftmargin \hskip\columnwidth}%
 \MakeFramed {\advance\hsize-\width
   \@totalleftmargin\z@ \linewidth\hsize
   \@setminipage}}%
 {\par\unskip\endMakeFramed%
 \at@end@of@kframe}
\makeatother

\makeatletter
\@ifundefined{Shaded}{
}{\renewenvironment{Shaded}{\begin{kframe}}{\end{kframe}}}
\@ifpackageloaded{fancyvrb}{%
  % https://github.com/CTeX-org/ctex-kit/issues/331
  \RecustomVerbatimEnvironment{Highlighting}{Verbatim}{commandchars=\\\{\},formatcom=\xeCJKVerbAddon}%
}{}
\makeatother

\usepackage{makeidx}
\makeindex

\urlstyle{tt}

\usepackage{amsthm}
\makeatletter
\def\thm@space@setup{%
  \thm@preskip=8pt plus 2pt minus 4pt
  \thm@postskip=\thm@preskip
}
\makeatother

\frontmatter

\title{Face lab book}
\author{Face lab}
\date{2024-11-29}

\usepackage{amsthm}
\newtheorem{theorem}{Theorem}[chapter]
\newtheorem{lemma}{Lemma}[chapter]
\newtheorem{corollary}{Corollary}[chapter]
\newtheorem{proposition}{Proposition}[chapter]
\newtheorem{conjecture}{Conjecture}[chapter]
\theoremstyle{definition}
\newtheorem{definition}{Definition}[chapter]
\theoremstyle{definition}
\newtheorem{example}{Example}[chapter]
\theoremstyle{definition}
\newtheorem{exercise}{Exercise}[chapter]
\theoremstyle{definition}
\newtheorem{hypothesis}{Hypothesis}[chapter]
\theoremstyle{remark}
\newtheorem*{remark}{Remark}
\newtheorem*{solution}{Solution}
\begin{document}
\maketitle


\thispagestyle{empty}

\begin{center}
献给……

\end{center}

\setlength{\abovedisplayskip}{-5pt}
\setlength{\abovedisplayshortskip}{-5pt}

{
\setcounter{tocdepth}{2}
\tableofcontents
}
\listoftables
\listoffigures
\chapter{简介}\label{ux7b80ux4ecb}

这是面孔实验室的实验手册,目前主要尝试介绍一些常用的软件或技术。

实验室目前常用的软件有:

\begin{itemize}
\tightlist
\item
  第 \ref{jamovi} 章: Jamovi\\
\item
  第 \ref{zotero} 章: Zotero
\item
  第 \ref{vscode} 章: Visual Studio Code
\end{itemize}

\mainmatter

\part{统计软件}\label{part-ux7edfux8ba1ux8f6fux4ef6}

\chapter{Jamovi}\label{jamovi}

汇总:沈佳欣、张宇杰\\
更新于:2024-11-07

\section{方差分析}\label{ux65b9ux5deeux5206ux6790}

\subsection{重复测量方差分析}\label{ux91cdux590dux6d4bux91cfux65b9ux5deeux5206ux6790}

1.导入数据后,依次点击``分析''\,``方差分析''选项,然后在弹出的选项栏中选择适当的方差分析方法。前五个为参数检验方法,后两个为非参数检验方法,它们分别对应统计课本中的克-瓦氏单向方差分析和弗里德曼两因素等级方差分析。

\begin{figure}

{\centering \includegraphics[width=0.8\linewidth]{img/jamovi/anova} 

}

\caption{方差分析}\label{fig:jamovi-anova}
\end{figure}

2.此次以重复测量方差分析为例。在下图左侧的界面里,左侧是我们数据的名称,右侧的``重复测量因子''是指在研究中设置的``被试内自变量''。在本例中,我们有三个``被试内自变量'',每个自变量都有两个水平。

\begin{figure}

{\centering \includegraphics[width=0.6\linewidth]{img/jamovi/rmanova-factor} 

}

\caption{自变量}\label{fig:jamovi-rmanova-factor}
\end{figure}

3.例如我们将``重复测量因子1''定义为Familiar,它包括``familiar''和``unfamiliar''两个水平。

\begin{figure}

{\centering \includegraphics[width=0.4\linewidth]{img/jamovi/rmanova-factorlevels} 

}

\caption{自变量水平}\label{fig:jamovi-rmanova-factorlevels}
\end{figure}

4.定义好重复测量因子后,``重复测量单元''中会出现各个自变量水平的组合,而``重复测量单元''也就是指研究中的各种实验条件。在本例中我们的实验是一个2 * 2 * 2的被试内设计,所以共有8种实验条件,我们需要做的是将数据拖入其对应的重复测量单元中。

\begin{figure}

{\centering \includegraphics[width=0.6\linewidth]{img/jamovi/rmanova-factorlevels2} 

}

\caption{自变量水平2}\label{fig:jamovi-rmanova-factorlevels2}
\end{figure}

5.接下来我们要选择效应量为``偏η²值'',以及确定因变量标签。本例中因变量标签为``d-prime''.

\begin{figure}

{\centering \includegraphics[width=0.6\linewidth]{img/jamovi/rmanova-dv} 

}

\caption{效应量偏η²}\label{fig:jamovi-rmanova-dv}
\end{figure}

6.下面的五个选项里,我们应当关注的主要是``适用条件判断''和``事后检验''。其中``事后检验''在之后的``2.2简单效应分析''一节中介绍。本节中我们先关注``适用条件判断''。

\begin{figure}

{\centering \includegraphics[width=0.6\linewidth]{img/jamovi/rmanova-posthoc} 

}

\caption{五个选项}\label{fig:jamovi-rmanova-posthoc1}
\end{figure}

7.对本专业来说,在``适用条件判断''这一栏中较为重要的是``方差齐性检验'',如果变量中涉及被试间变量,那么应当进行方差齐性检验。但由于本例中的变量都是被试内变量,所以无需进行方差齐性检验。其他几个选项也可以根据研究所需选取。

\begin{figure}

{\centering \includegraphics[width=0.6\linewidth]{img/jamovi/rmanova-sphericity} 

}

\caption{适用条件判断}\label{fig:jamovi-rmanova-sphericity}
\end{figure}

8.所有设定完成后,Jamovi会在右侧呈现数据分析结果。

\begin{figure}

{\centering \includegraphics[width=1\linewidth]{img/jamovi/rmanova-result} 

}

\caption{方差分析结果}\label{fig:jamovi-rmanova-result}
\end{figure}

\section{简单效应分析}\label{ux7b80ux5355ux6548ux5e94ux5206ux6790}

如果F检验的结果表明差异不显著,说明实验中的自变量对因变量没有显著影响。相反,如果方差分析下检验的结果表明差异显著,拒绝了虚无假设,就表明几个实验处理组的两两比较中至少有一对平均数间的差异达到了显著水平,至于是哪一对,方差分析并没有回答。虚无假设被拒绝的结果一旦出现,就必须对各实验处理组的多对平均数进一步分析做深入比较,判断究竟是哪一对或哪几对的差异显著,哪几对不显著,确定两变量关系的本质,这就是事后检验(post hoc test)。这个统计分析过程也被称作事后多重比较(multiple comparison procedures)。

\subsection{重复测量方差分析中的简单效应分析}\label{ux91cdux590dux6d4bux91cfux65b9ux5deeux5206ux6790ux4e2dux7684ux7b80ux5355ux6548ux5e94ux5206ux6790}

这一部分承接上述重复测量方差分析。在本例中,当我们在Jamovi中确定好了``重复测量因子''即``被试内变量''和``重复测量单元''即``实验条件''后,我们可以看到右侧的结果栏中就可以呈现统计结果,如果某两个或几个变量之间存在显著的交互作用,那么我们应当进行事后检验,以确定一个变量在与它有交互作用的另一个变量的哪些水平上显著,哪些水平上不显著。

1.我们需要先选择下列五个选项中的``事后检验''。再选择需要进行事后检验的交互变量,并按照研究需要选择恰当的校正方法。

\begin{figure}

{\centering \includegraphics[width=0.6\linewidth]{img/jamovi/rmanova-posthoc} 

}

\caption{事后检验}\label{fig:jamovi-rmanova-posthoc2}
\end{figure}

2.右侧窗口会生成事后检验的结果,其中最右边两列就是Jamovi按照我们选择的``Tukey法''和``Bonferron法''计算出的统计量。

\begin{figure}

{\centering \includegraphics[width=1\linewidth]{img/jamovi/rmanova-posthoc-results} 

}

\caption{事后检验结果}\label{fig:jamovi-rmanova-posthoc-results}
\end{figure}

\section{T检验}\label{tux68c0ux9a8c}

\begin{itemize}
\tightlist
\item
  首先,导入数据
\item
  其次,点击选项栏的分析,选择对应的T检验。
\end{itemize}

\begin{figure}

{\centering \includegraphics[width=0.5\linewidth]{img/jamovi/ttest} 

}

\caption{T检验}\label{fig:jamovi-ttest}
\end{figure}

\subsection{配对样本T检验}\label{ux914dux5bf9ux6837ux672ctux68c0ux9a8c}

本例中介绍``配对样本T检验''。接着,选择要分析的变量

\begin{figure}

{\centering \includegraphics[width=0.6\linewidth]{img/jamovi/ttest-compare} 

}

\caption{变量选择}\label{fig:jamovi-ttest-compare}
\end{figure}

\begin{itemize}
\tightlist
\item
  然后,可以按需勾选,描述就是分析数据的均值、标准差、标准误等描述性数据;描述图就是生成图表
\end{itemize}

\begin{figure}

{\centering \includegraphics[width=0.8\linewidth]{img/jamovi/ttest-results} 

}

\caption{配对样本T检验结果}\label{fig:jamovi-ttest-results}
\end{figure}

\section{额外模块推荐}\label{ux989dux5916ux6a21ux5757ux63a8ux8350}

\begin{figure}

{\centering \includegraphics[width=1\linewidth]{img/jamovi/modules} 

}

\caption{添加额外模块}\label{fig:jamovi-modules}
\end{figure}

\begin{figure}

{\centering \includegraphics[width=1\linewidth]{img/jamovi/modules-library} 

}

\caption{额外模块库}\label{fig:jamovi-modules-library}
\end{figure}

可以通过jamovi官网下载自己所需要的库。这是网址:\url{https://www.jamovi.org/library.html}

\begin{figure}

{\centering \includegraphics[width=1\linewidth]{img/jamovi/modules1} 

}

\caption{额外模块}\label{fig:jamovi-modules1}
\end{figure}

\begin{figure}

{\centering \includegraphics[width=1\linewidth]{img/jamovi/modules2} 

}

\caption{额外模块1}\label{fig:jamovi-modules2}
\end{figure}

\begin{figure}

{\centering \includegraphics[width=1\linewidth]{img/jamovi/modules3} 

}

\caption{额外模块2}\label{fig:jamovi-modules3}
\end{figure}

\begin{figure}

{\centering \includegraphics[width=1\linewidth]{img/jamovi/modules4} 

}

\caption{额外模块3}\label{fig:jamovi-modules4}
\end{figure}

\chapter{R语言}\label{r}

\section{预处理数据}\label{r-prepro}

汇总:
更新于:

\section{重复测量方差分析}\label{r-rm-anova}

汇总:
更新于:

\section{数据可视化}\label{r-plot}

汇总:
更新于:

\part{科研工具}\label{part-ux79d1ux7814ux5de5ux5177}

\chapter{Zotero}\label{zotero}

汇总:郭沫然\\
更新于:

\part{其他工具}\label{part-ux5176ux4ed6ux5de5ux5177}

\chapter{Visual Studio Code}\label{vscode}

\section{下载安装}\label{ux4e0bux8f7dux5b89ux88c5}

请在\href{https://code.visualstudio.com/}{Visual Studio Code官网}下载该软件。

\section{制作公开数据的README.md文件}\label{mkreadme}

更新于:2024-11-29

\begin{enumerate}
\def\labelenumi{\arabic{enumi}.}
\tightlist
\item
  在Visual Studio Code中新建一个文本文档:
\end{enumerate}

\begin{figure}

{\centering \includegraphics[width=0.6\linewidth]{img/vscode/mkreadme_mknewfile} 

}

\caption{在VS Code中新建一个文本文档}\label{fig:mkreadme-mknewfile}
\end{figure}

之后,你应该看到类似的新建文件:

\begin{figure}

{\centering \includegraphics[width=1\linewidth]{img/vscode/mkreadme_newfile} 

}

\caption{VS Code中的新建文本文档}\label{fig:mkreadme-newfile}
\end{figure}

2. 将新建文档保存为\texttt{README.md}文件:

\begin{figure}

{\centering \includegraphics[width=0.7\linewidth]{img/vscode/mkreadme_saveas1} 

}

\caption{将新建文本文档另存为...}\label{fig:mkreadme-saveas1}
\end{figure}

在打开的对话框中,请选择保存新建文本文档的位置,并将文件命名为\texttt{README.md}:

\begin{figure}

{\centering \includegraphics[width=0.8\linewidth]{img/vscode/mkreadme_saveas2} 

}

\caption{重命名文件为“README.md”并选择文件存储位置}\label{fig:mkreadme-saveas2}
\end{figure}

3. 根据数据文件编辑\texttt{README.md}文件

在\texttt{README.md}文件中,你可以添加根据数据文件的内容,添加列名称(\texttt{Header})和内容描述(\texttt{Description}):

\begin{Shaded}
\begin{Highlighting}[]

\NormalTok{Header | Description}
\NormalTok{{-}{-}{-}|{-}{-}{-}}
\NormalTok{Column1 | Description of Column1}
\NormalTok{Column2 | Description of Column2}
\end{Highlighting}
\end{Shaded}

假定我们用有这样一个数据文件:

\begin{figure}

{\centering \includegraphics[width=1\linewidth]{img/vscode/mkreadme_datafile} 

}

\caption{数据文件}\label{fig:mkreadme-datafile}
\end{figure}

我们相应地编辑\texttt{README.md}文件:

\begin{Shaded}
\begin{Highlighting}[]

\NormalTok{Header | Description}
\NormalTok{{-}{-}{-}|{-}{-}{-}}
\NormalTok{SubjCode | Subject Code}
\NormalTok{Trial Number | trial number}
\NormalTok{Orientation | orientation of the stimulus: }\InformationTok{\textasciigrave{}upr\textasciigrave{}}\NormalTok{: upright vs }\InformationTok{\textasciigrave{}inv\textasciigrave{}}\NormalTok{: inverted}
\NormalTok{Congruency | congruency of the trial type in complete composite task: }\InformationTok{\textasciigrave{}con\textasciigrave{}}\NormalTok{: congruent vs }\InformationTok{\textasciigrave{}inc\textasciigrave{}}\NormalTok{: incongruent}
\NormalTok{Alignment | whether the top and bottom halves of faces are aligned (}\InformationTok{\textasciigrave{}ali\textasciigrave{}}\NormalTok{) or misaligned (}\InformationTok{\textasciigrave{}mis\textasciigrave{}}\NormalTok{)}
\NormalTok{CorrectAnswer | whether the correct response was the same (}\InformationTok{\textasciigrave{}s\textasciigrave{}}\NormalTok{) or different (}\InformationTok{\textasciigrave{}d\textasciigrave{}}\NormalTok{)}
\NormalTok{Acc | whether the response was correct (}\InformationTok{\textasciigrave{}1\textasciigrave{}}\NormalTok{) or incorrect (}\InformationTok{\textasciigrave{}0\textasciigrave{}}\NormalTok{)}
\NormalTok{isSame | whether participant reported the same (}\InformationTok{\textasciigrave{}1\textasciigrave{}}\NormalTok{) or not (}\InformationTok{\textasciigrave{}0\textasciigrave{}}\NormalTok{) for that trial}
\NormalTok{RT | response time in milliseconds}
\end{Highlighting}
\end{Shaded}

4. 编辑\texttt{README.md}文件的同时预览效果

在编辑\texttt{README.md}文件的同时,你可以通过以下方式来预览效果:

\begin{itemize}
\tightlist
\item
  \texttt{Ctrl\ +\ Shift\ +\ V} (Windows)快捷键
\item
  \texttt{Cmd\ +\ Shift\ +\ V} (Mac)快捷键
\item
  右键点击\texttt{README.md}文件,选择\texttt{Open\ Preview}选项
\end{itemize}

\begin{figure}

{\centering \includegraphics[width=0.6\linewidth]{img/vscode/mkreadme_openpreview} 

}

\caption{右键点击`README.md`文件,选择`Open Preview`选项}\label{fig:mkreadme-openpreview}
\end{figure}

预览效果与下图类似:

\begin{figure}

{\centering \includegraphics[width=1\linewidth]{img/vscode/mkreadme_preview} 

}

\caption{预览md文件}\label{fig:mkreadme-preview}
\end{figure}

5. 将\texttt{README.md}文件保存为pdf文件

完成编辑\texttt{README.md}文件后,我们可以将其保存为或生成一个pdf文档。

5.1 在VS Code中安装\texttt{Markdown\ PDF}插件

\begin{figure}

{\centering \includegraphics[width=0.4\linewidth]{img/vscode/mkreadme_extension} 

}

\caption{VS Code中打开Extension}\label{fig:mkreadme-extension}
\end{figure}

在搜索框中输入\texttt{Markdown\ PDF},并点击安装(\texttt{Install}):

\begin{figure}

{\centering \includegraphics[width=0.5\linewidth]{img/vscode/mkreadme_markdownpdf} 

}

\caption{搜索并安装Markdown PDF}\label{fig:mkreadme-markdownpdf}
\end{figure}

5.2 生成pdf文件

安装完成\texttt{Markdown\ PDF}插件后,请确保\texttt{README.md}已在VS Code中打开,然后通过\texttt{View\ -\textgreater{}\ Command\ Palette}打开\texttt{Command\ Palette}:

\begin{figure}

{\centering \includegraphics[width=0.75\linewidth]{img/vscode/mkreadme_palette} 

}

\caption{打开Command Palette}\label{fig:mkreadme-palette}
\end{figure}

然后在\texttt{Command\ Palette}中搜索并打开\texttt{Markdown\ PDF:\ Export\ (pdf)}:

\begin{figure}

{\centering \includegraphics[width=0.6\linewidth]{img/vscode/mkreadme_exportpdf} 

}

\caption{Markdown PDF Export (pdf)}\label{fig:mkreadme-exportpdf}
\end{figure}

之后,你应该在VS Code右下角看到正在生成pdf文件的提示消息:

\begin{figure}

{\centering \includegraphics[width=0.8\linewidth]{img/vscode/mkreadme_exportmessage} 

}

\caption{正在生成PDF文件...}\label{fig:mkreadme-exportmessage}
\end{figure}

提示消息消失后,你可以在\texttt{README.md}文件所在的文件夹中找到生成的pdf文件:

\begin{figure}

{\centering \includegraphics[width=0.6\linewidth]{img/vscode/mkreadme_mkpdf} 

}

\caption{正在生成PDF文件...}\label{fig:mkreadme-mkpdf}
\end{figure}

生成的pdf文件效果如下:

\begin{figure}

{\centering \includegraphics[width=1\linewidth]{img/vscode/mkreadme_pdf} 

}

\caption{PDF文件}\label{fig:mkreadme-pdf}
\end{figure}

\part{实验材料和公开数据}\label{part-ux5b9eux9a8cux6750ux6599ux548cux516cux5f00ux6570ux636e}

\chapter{实验材料}\label{materials}

汇总:\\
更新于:

\section{Chicago Face Database (芝加哥面孔数据库)}\label{chicago-face-database-ux829dux52a0ux54e5ux9762ux5b54ux6570ux636eux5e93}

\href{https://www.chicagofaces.org/}{Chicago Face Database}

Ma, D. S., Correll, J., \& Wittenbrink, B. (2015). The Chicago face database: A free stimulus set of faces and norming data. \emph{Behavior Research Methods}, 47(4), 1122--1135. \url{https://doi.org/10.3758/s13428-014-0532-5}

Ma, D. S., Kantner, J., \& Wittenbrink, B. (2021). Chicago Face Database: Multiracial expansion. \emph{Behavior Research Methods}, 53(3), 1289--1300. \url{https://doi.org/10.3758/s13428-020-01482-5}

\chapter{公开数据库}\label{opendata}

汇总:\\
更新于:

\section{磁共振(fMRI)}\label{ux78c1ux5171ux632ffmri}

\section{脑电(EEG)}\label{ux8111ux7535eeg}

\section{行为}\label{ux884cux4e3a}

\cleardoublepage

\appendix \addcontentsline{toc}{chapter}{\appendixname}


\chapter{如何为本手册作出贡献}\label{contribute}

如果你发现这个手册对你有帮助,欢迎你也贡献一份力量。如果你觉得这个手册没有什么用处,同样欢迎你提出建议,和我们一起来完善它。

\section{直接Pull Request}\label{ux76f4ux63a5pull-request}

如果你已经可以非常熟练地使用 Git 和 GitHub,那么你可以直接在 GitHub 上提交issues或Pull Request。

\section{简易教程}\label{ux7b80ux6613ux6559ux7a0b}

如果你对 Git 和 GitHub 不太熟悉,也请不到担心,你仍然可以通过本教程,学习如何完成一个可以成为本手册一部分的Rmd文件。完成后你只需将新创建的Rmd文档发送给我们,审核通过后该文档内容就会出现在\href{https://haiyangjin.github.io/labbook/}{本实验室手册}。

\subsection{下载示例}\label{ux4e0bux8f7dux793aux4f8b}

首先请下载\href{https://github.com/HaiyangJin/labbook/archive/refs/heads/mini.zip}{示例}。解压后你会看到一个名为\texttt{labbook-mini}的文件夹。

\begin{figure}

{\centering \includegraphics[width=0.7\linewidth]{img/contribute/mini_dir} 

}

\caption{The minimal example directory}\label{fig:contri-mini-dir}
\end{figure}

\subsection{安装R和RStudio}\label{ux5b89ux88c5rux548crstudio}

如果你还没有安装R和RStudio,请先根据\href{https://haiyangjin.github.io/rpsych-cn/intro.html\#prepare}{这里的信息}安装相关软件以及进行相应的设置。

安装完成后,请不要忘了\href{https://haiyangjin.github.io/rpsych-cn/intro.html\#mirror}{设置R包的镜像源}。

完成后请安装bookdown包,可以在RStudio中运行以下代码来实现:

\begin{Shaded}
\begin{Highlighting}[]
\FunctionTok{install.packages}\NormalTok{(}\StringTok{"bookdown"}\NormalTok{)}
\end{Highlighting}
\end{Shaded}

如果遇到问题,可以尝试查看这里的\href{https://haiyangjin.github.io/rpsych-cn/qacn.html\#qacn}{帮助信息}。

\subsection{打开labbook.Rproj}\label{ux6253ux5f00labbook.rproj}

请\textbf{直接}双击\texttt{labbook-mini}文件夹中的\texttt{labbook.Rproj}文件,这样你就可以在RStudio中打开这个项目。

\begin{quote}
请\textbf{直接}双击\texttt{labbook-mini}文件夹中的\texttt{labbook.Rproj}文件,这样你就可以在RStudio中打开这个项目。
\end{quote}

\begin{theorem}
请\textbf{直接}双击\texttt{labbook-mini}文件夹中的\texttt{labbook.Rproj}文件,这样你就可以在RStudio中打开这个项目。
\end{theorem}

打开后你会看到如下的界面:

\begin{figure}

{\centering \includegraphics[width=0.9\linewidth]{img/contribute/mini_rproj} 

}

\caption{The minimal example R project}\label{fig:contri-mini-rproj}
\end{figure}

\textcolor{red}{\textbf{特别注意}}:请确保如上图 \ref{fig:contri-mini-rproj}右上角所示,当前的项目名称是\texttt{labbook-mini}。如果不是,请点击\texttt{File} -\textgreater{} \texttt{Open\ project...},然后选择\texttt{labbook-mini}文件夹中的\texttt{labbook.Rproj}文件,重新打开该项目。

\subsection{确认示例可以运行}\label{ux786eux8ba4ux793aux4f8bux53efux4ee5ux8fd0ux884c}

现在我们可以尝试运行这个示例了。请在RStudio右上角找到\texttt{Build},然后点击\texttt{Build\ Book}。如果一切顺利,会跳出一个窗口,同时你应该会在RStudio的右上角看到:

\begin{figure}

{\centering \includegraphics[width=0.7\linewidth]{img/contribute/mini_build} 

}

\caption{Build book for the minimal example}\label{fig:contri-mini-build}
\end{figure}

如果你不小心关掉了跳出的窗口,你应该会看到在\texttt{labbook-mini}文件夹中生成了一个\texttt{\_book}子文件夹,里面包含了一个\texttt{index.html}文件。请双击这个文件通过浏览器打开,你会看到一个网页版的示例。

\subsection{创建一个新的Rmd文件}\label{ux521bux5efaux4e00ux4e2aux65b0ux7684rmdux6587ux4ef6}

现在我们可以开始创建一个新的Rmd文件来介绍你想贡献的内容了。比如你可以打开\texttt{20-new.Rmd}文件,然后将其另存为\texttt{66-your\_topic.Rmd}。请确保你的文件名中包含了你的主题,这样我们可以更容易地了解不同\texttt{Rmd}文件的内容。

例如,你可以输入这些内容:

\begin{Shaded}
\begin{Highlighting}[]
\FunctionTok{\# 我的主题}

\InformationTok{    汇总:最棒的我     }
\InformationTok{    更新于:很开心的今天(或者2024年11月11日)}

\InformationTok{    就是想分享一下今天的快乐。。。}

\FunctionTok{\#\# 加一个二级标题}


\FunctionTok{\#\#\# 需要放一张照片}

\SpecialStringTok{{-} }\NormalTok{添加图片方式一:}
\InformationTok{\textasciigrave{}\textasciigrave{}\textasciigrave{}\{r test{-}my{-}topic, echo=FALSE, fig.cap=\textquotesingle{}我的图片\textquotesingle{}, out.width=\textquotesingle{}60\%\textquotesingle{}, fig.align=\textquotesingle{}center\textquotesingle{}\}}
\InformationTok{knitr::include\_graphics(file.path("img", "jamovi", "anova.png"))}
\InformationTok{\textasciigrave{}\textasciigrave{}\textasciigrave{}}

\SpecialStringTok{{-} }\NormalTok{添加图片方式二:}
\AlertTok{![](img/jamovi/modules{-}library.png)}

\FunctionTok{\#\# 再给文字加一些颜料}

\NormalTok{随随便便放一些文字,看看效果如何。}

\NormalTok{我还想再给某一些文字加个链接,比如}\CommentTok{[}\OtherTok{这里}\CommentTok{](https://www.jamovi.org/)}\NormalTok{。}
\NormalTok{或者再加一些特定的颜色,比如}\CommentTok{[}\OtherTok{红色}\CommentTok{]}\NormalTok{\{color="red"\}和}\CommentTok{[}\OtherTok{蓝色}\CommentTok{]}\NormalTok{\{color="green"\},}
\CommentTok{[}\OtherTok{对的,}\CommentTok{]}\NormalTok{\{color="\#5bc0eb"\}}\CommentTok{[}\OtherTok{我就是}\CommentTok{]}\NormalTok{\{color="\#fa7921"\}}\CommentTok{[}\OtherTok{想弄个}\CommentTok{]}\NormalTok{\{color="\#9bc53d"\}}
\CommentTok{[}\OtherTok{Stroop效应}\CommentTok{]}\NormalTok{\{color="\#e55934"\}(抱歉,颜色有点丑)。}

\FunctionTok{\#\#\# 再来一个列表}

\SpecialStringTok{{-} }\NormalTok{一个}
\SpecialStringTok{{-} }\NormalTok{两个}
\SpecialStringTok{{-} }\NormalTok{三个}

\SpecialStringTok{1. }\NormalTok{第一个}
\SpecialStringTok{2. }\NormalTok{第二个}
\SpecialStringTok{3. }\NormalTok{第三个}
\end{Highlighting}
\end{Shaded}

\subsection{重新Build book}\label{ux91cdux65b0build-book}

输入一些Rmarkdown内容后,请通过RStudio右上角的\texttt{Build},再一次\texttt{Build\ Book}及时检查是否可以成功生成网页文件。\textcolor{red}{如果你已经输入了上述所有代码,那你应该会看到}:

\begin{figure}

{\centering \includegraphics[width=0.8\linewidth]{img/contribute/new_rmd_output1} 

}

\caption{The new Rmd file output}\label{fig:contri-new-rmd-1}
\end{figure}
\begin{figure}

{\centering \includegraphics[width=0.8\linewidth]{img/contribute/new_rmd_output2} 

}

\caption{The new Rmd file output}\label{fig:contri-new-rmd-2}
\end{figure}

\subsection{其他}\label{ux5176ux4ed6}

如果你想再了解其他关于Rmarkdown/markdown的一些``特殊''用法,可以看看\href{https://bookdown.org/yihui/bookdown/markdown-syntax.html\#block-level-elements}{这里}。

就先这样吧。如果你有任何问题,欢迎随时联系\href{https://haiyangjin.github.io/en/contact/}{我们}。

期待着你的Rmd会出现在手册里!

\chapter{常见问题}\label{qa}

\section{R和RStudio相关问题}\label{rux548crstudioux76f8ux5173ux95eeux9898}

有关R和RStudio的相关问题请查看\href{https://haiyangjin.github.io/rpsych-cn/qacn.html\#qacn}{这里}。

\section{格式调整}\label{ux683cux5f0fux8c03ux6574}

\subsection{文字}\label{ux6587ux5b57}

\subsubsection{文字颜色}\label{ux6587ux5b57ux989cux8272}

你可以通过以下代码设置文字的颜色:

\begin{Shaded}
\begin{Highlighting}[]
\CommentTok{[}\OtherTok{红色}\CommentTok{]}\NormalTok{\{color="red"\}   }
\CommentTok{[}\OtherTok{蓝色}\CommentTok{]}\NormalTok{\{color="green"\}  }
\CommentTok{[}\OtherTok{绿色}\CommentTok{]}\NormalTok{\{color="blue"\}  }
\CommentTok{[}\OtherTok{黄色}\CommentTok{]}\NormalTok{\{color="yellow"\}}
\end{Highlighting}
\end{Shaded}

相应的显示为:\\
\textcolor{red}{红色}\\
\textcolor{green}{蓝色}\\
\textcolor{blue}{绿色}\\
\textcolor{yellow}{黄色}

\subsection{图片}\label{ux56feux7247}

强烈建议使用以下格式显示图片,这样可以灵活的设置图片的大小和位置等。

\begin{Shaded}
\begin{Highlighting}[]
\InformationTok{\textasciigrave{}\textasciigrave{}\textasciigrave{}\{r test{-}my{-}topic, echo=FALSE, fig.cap=\textquotesingle{}我的图片\textquotesingle{}, out.width=\textquotesingle{}60\%\textquotesingle{}, fig.align=\textquotesingle{}center\textquotesingle{}\}}
\InformationTok{knitr::include\_graphics(file.path("img", "jamovi", "anova.png"))}
\InformationTok{\textasciigrave{}\textasciigrave{}\textasciigrave{}}
\end{Highlighting}
\end{Shaded}

\begin{figure}

{\centering \includegraphics[width=0.7\linewidth]{img/contribute/mini_dir} 

}

\caption{这是描述图片的题目}\label{fig:qa-img-run}
\end{figure}

\begin{itemize}
\tightlist
\item
  \texttt{qa-img}:图片的名称,可以自定义(但请不能在同一个文档中重复)。该名称也可以通过 \texttt{\textbackslash{}@ref(fig:qa-img)} 引用该图片,如\texttt{\textbackslash{}@ref(fig:qa-img)}。
\item
  \texttt{echo=TRUE}:是否显示该 R chunk 代码。一般情况下应设为 \texttt{FALSE}。这里为了展示代码,所以设为 \texttt{TRUE}。
\item
  \texttt{fig.cap=\textquotesingle{}这是描述图片的题目\textquotesingle{}}:图片的题目。
\item
  \texttt{out.width=\textquotesingle{}70\%\textquotesingle{}}:图片的宽度,可以设置为百分比或者像素。
\item
  \texttt{fig.align=\textquotesingle{}center\textquotesingle{}}:图片的位置,可以设置为 \texttt{center}、\texttt{left} 或者 \texttt{right}。
\end{itemize}

\bibliography{references.bib,packages.bib}

\backmatter
\printindex

\end{document}
